\documentclass[11pt]{article}
\usepackage{amsmath,amssymb,amsthm}
\usepackage{algorithm}
\usepackage[noend]{algpseudocode} 

%---enable russian----

\usepackage[utf8]{inputenc}
\usepackage[russian]{babel}


%---tikz----
\usepackage{tikz}
\usetikzlibrary{arrows, chains, matrix, positioning, scopes, patterns, shapes}

%----newcommands (math)----
\input{header} 


%-----newcommands specific to the scribe template 
\newcommand{\handout}[5]{
  \noindent
  \begin{center}
  \framebox{
    \vbox{
      \hbox to 5.78in { {\bf  } \hfill #2 }
      \vspace{4mm}
      \hbox to 5.78in { {\Large \hfill #5  \hfill} }
      \vspace{2mm}
      \hbox to 5.78in { {\em #3 \hfill #4} }
    }
  }
  \end{center}
  \vspace*{4mm}
}

\newcommand{\lecture}[4]{\handout{#1}{#2}{#3}{Выполнил(а): #4}{#1}}


\newtheorem{theorem}{Теорерма}
\newtheorem{corollary}[theorem]{Следствие}
\newtheorem{lemma}[theorem]{Лемма}
\newtheorem{observation}[theorem]{Observation}
\newtheorem{proposition}[theorem]{Предложение}
\newtheorem{definition}[theorem]{Определение}
\newtheorem{claim}[theorem]{Утверждение}
\newtheorem{fact}[theorem]{Факт}
\newtheorem{assumption}[theorem]{Предположение}

% 1-inch margins
\topmargin 0pt
\advance \topmargin by -\headheight
\advance \topmargin by -\headsep
\textheight 8.9in
\oddsidemargin 0pt
\evensidemargin \oddsidemargin
\marginparwidth 0.5in
\textwidth 6.5in

\parindent 0in
\parskip 1.5ex

\begin{document}

\lecture{Название Лекции}{\today}{Лектор: Елена Киршанова}{ВАШЕ ИМЯ}

\section{Название первого раздела}

Правила оформления лекции:

\begin{enumerate}
	\item Оформляются лекции в соответствии с этим шаблоном
	\item Добавлять новый макроса через \verb|newcommand| следует в файл \texttt{header.tex}
	\item Оформление лемм, теорем, нумерованных и ненумерованных список. алгоритмов, рисунков, списка литераторы следует выполнять так, как это сделано в этом файле
\end{enumerate}


\begin{theorem} \label{thm:nameOfTheTheorem}
	Утверждение теоремы
\end{theorem}


\begin{proof}
	Доказательство
\end{proof}

Ссылка на Теорему~\ref{thm:nameOfTheTheorem} оформляется с помощью команды \verb|ref|. Аналогично для лемм, следствий и т.д.


Короткую формулу можно включить в текст, например,  $\log_2{4} = 2$. Более громоздкую формулу лучше оформить на новой строке с помощью оператора \verb+\[+\dots\verb+\]+:
\[
\int_{0}^{\infty} \frac{x}{e^x - 1 } dx = \frac{\pi^2}{6}
\]

Многострочные ненумерованный формулы оформляются с помощью окружения \verb+\begin{align}*+
\begin{align*}
(x+y)^2 &= x^2 + 2xy + y^2 \\
F&=  G \frac{m_1 m_2}{d^2}
\end{align*}

Нумерованные формулы оформляются либо через \verb+align+, либо через \verb+equation+:
\begin{equation} \label{eq:potential_energy}
E_p = mgh
\end{equation}

Ссылка на Уравнение~\eqref{eq:potential_energy} выполняется с помощью \verb+\eqref{}+ (в случае использования \verb+equation+), либо с помощью \verb+(\ref{})+ в случае использования  \verb+align+. Ссылка задаётся оператором  \verb+label{}+. 


\section{Оформление алгоритма}



\begin{algorithm}[h]
	\caption{Алгоритм Евклида}\label{alg:euclid}
	\begin{algorithmic}[1]
		\Require $a,b \in \mathbb{Z}$
		\Ensure $\text{НОД}(a,b)$
		\State $r\gets a\bmod b$
		\While{$r\not=0$} 
		\State $a\gets b$
		\State $b\gets r$
		\State $r\gets a\bmod b$
		\EndWhile 
		\State Return $b$
	\end{algorithmic}

\end{algorithm}


\section{Анализ}
\subsection{Blah blah blah}
Текст лекции

\subsubsection{Blah blah blah}
Текст лекции




\paragraph{Литература.}
Ссылки на литературу, упомянутую в лекции. Если используется BibTeX, необходимо  .bbl file into your .tex
source. 

\end{document}